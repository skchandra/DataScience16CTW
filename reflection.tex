\documentclass[10pt]{article}
\usepackage[margin=1in,twocolumn]{geometry}
\usepackage[utf8]{inputenc}
\usepackage{graphicx}
\usepackage{placeins}
\usepackage{amssymb}
\usepackage{amsmath}
\usepackage{epstopdf}

\newcommand{\degree}{$^{\circ}$}    % degree symbol

\title{Sexual Assault and Immigration:\\
		\Large About Immigrants in Sweden \\
		\Large Final Reflection}
\author{Kiki Chandra and Pratool Gadtaula}
\date{March 11, 2016}

\begin{document}

\maketitle

\section{Assessment Evidence and Interpretation}
	Our assessment plan involved being assessed based on the thoroughness and journalistic nature of our insights. We also wanted to be assessed on how successfully we keep our own bias from getting in the way of what the data presents, and how meaningful and compelling our visualizations are. Our primary piece of evidence is, of course, our article (available at http://skchandra.github.io/DataScience16CTW/).\\
	Through our article, we tried to present an overview of the issue at hand, as well as some existent factors that contribute to the problems. We have a strong bias but tried to talk about both sides of the issue when interpreting data and openly acknowledge that our explanations are not fact. Our thoroughness can also be demonstrated through our ipython notebooks, where we tried multiple manipulations of the data before deciding on which facts to mention and which figures to include. 
	
\section{Changing the World}
Our project may not have the potential to change the world directly or on an enormous scale, but we think it has the potential to start a conversation around immigration that is more geared towards exploring socio-economic factors and the impact they have rather than simply blaming immigrants without context. By introducing some other factors that should be considered when looking at the media’s representation of immigrants and crime in Sweden, we have hopefully been able to show that there are more factors at play than just ethnicity. If we wanted to have a larger, more concrete impact, more data would be needed, but Swedish agencies no longer collect or publish breakdowns of statistics by ethnicity, so this would be difficult to find. 

\section{Learning Goals}
We think we were very successful in completing our learning objectives. We met around three times a week mainly to work and determine what parts we each wanted to do along with research for data. While we were not expecting this, in addition to our learning goals, we learned to tap into the vast swaths of information on the Internet for reliable statistics and data. This was painful at times, as neither of us are literate in Swedish, but with the aid of online translation tools we were eventually able to get the most powerful pieces of data for contextualizing and persuading the reader. We also learned how to use interactive plotting tools, such as Plotly, D3, and Amcharts mainly on our own with a strong awareness of each other’s time capacity for this project (for instance if we had an exam the next day or another project due). Best of all we were able to bring together our individual research and collaboratively explore our interests. We did have one drawback, that was partially due to the time constraint of the project, in that we would have liked a more polished final product. This isn’t a large issue though, as we decided to finish polishing up the website and visualizations at a later point in time.

\end{document}