\documentclass[10pt]{article}
\usepackage[margin=1in,twocolumn]{geometry}
\usepackage[utf8]{inputenc}
\usepackage{graphicx}
\usepackage{placeins}
\usepackage{amssymb}
\usepackage{amsmath}
\usepackage{epstopdf}

\newcommand{\degree}{$^{\circ}$}    % degree symbol

\title{Sexual Assault and Immigration:\\
		\Large Misinterpreted Data and Lies\\
		\Large About Immigrants in Sweden}
\author{Kiki Chandra and Pratool Gadtaula}
\date{February 25, 2016}

\begin{document}

\maketitle

\section{Motivation}
	After looking into immigration with an initial intent of the variation of immigration laws over time all over the world, we came across the idea to reflect the type and frequency of crimes that immigrants commit or don’t commit as compared to the the rest of the non-immigrant population in that area. Sexual violence and immigration have been linked together for years in the media and politics in particularly negative ways. With the motivation to uncover the truth, we quickly uncovered the massive amount of visualizations of data surrounding sexual violence statistics in Sweden. Especially in the case of Sweden, sexual violence statistics have been misconstrued to try and prove that immigrants of certain backgrounds, in this case Muslims, are bringing problems that did not previously exist on a large scale. At a time when racial and religious tensions are rampant in society, it is important to work to combat unjustified prejudices and disprove harmful assumptions.
	
\section{Main Ideas}
	In order to change people’s minds and prove that biases against people exist and should be confronted instead of scapegoating. In order to change people’s minds, we want present a dramatic visualization that strikes the viewer with empathy and a sense of humanity. We imagine this manifesting in a visualization that represents the conclusions we were able to gather. This visualization would be part of a non-scientific article presented to a broad audience that describes and digs into our insights. As a stretch goal, we would like this visualization to be interactive and also tell a story - explaining people’s experiences and livelihoods not as a simple series of numbers.
	
\section{Learning Goals}
	We would mainly like to work on learning how to use data in a persuasive way, either to prove a point or bring awareness to a more overlooked problem. In doing this, it would also be really interesting to figure out how to go about structuring our project and determining what information we need. Additionally, we both want to learn how to work with very technical information and data and present it in a format for a more general audience.

\section{Data Sources}
	Most of our data is available through government statistics sites, which house information on immigration data and crime statistics through the years, starting from around the 1990’s to 2014. If we have time, it would also be interesting to see if we can incorporate anecdotal information from news articles to provide a framework for our work.
	
\section{Deliverables and Assessment}
	Our final deliverable as of now is going to be an article that depicts the story of the issue at hand, and our approach to disproving some common misassumptions through with a potentially interactive data visualization (or multiple visualizations). We will get started quickly by learning the basics of D3 (as we are very interested in making our final visualization interactive). Then we will mostly pair-program and understand and parse the data together in meetings. We may also try generating insights for our written report individually. We plan to have a solid start on figuring out D3 visualizations, and have collected all of our data. Hopefully, we will also have gotten to the point where we can plan out our article or even start writing. For assessment, we believe that we could be fairly assessed based on the thoroughness of our insights -- we expect it to be journalistic in nature. We also want to be assessed on our bias and the meaningfulness of our visualization(s).
	
\newpage

\section*{Sources}
\texttt{https://www.dhs.gov/data-statistics\\
https://www.bra.se/bra/bra-in-english/home/crime-and-statistics/crime-statistics.html }

\end{document}